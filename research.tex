\title{Existing Solutions and Our Ideas}

\documentclass[12pt]{article}

\begin{document}
\maketitle

\section{Problem Description}
Suppose you wish to create a recording of yourself as you move around and do activities. Usually, you have a middle man that is between you and your camera so that the they can control the camera so that it is continuously focused on you. However, it is not always the case that this cameraman is available to help you film yourself.

\section{Problem Solution}
To eliminate the need for a cameraman, a camera is first placed on a custom mount. Then from reading the input stream of the camera, software we track your motion displacement and control the mount to follow your motions. The mount is capable of rotating along a horizontal plane and a vertical plane.

\section{Existing Solution}
\subsection{SoloShot}
Soloshot solves the problem of not having a cameraman by having a custom mount that tracks user motions. A camera is then placed on top of the mount to film. The key to Soloshot’s solution is that the user must wear a tracking tag as they do their performance. The custom mount contains a component - called the base - that tracks the tag’s position via a real-time location system. This base tracks the tag and rotates accordingly using internal rotators.
\par
There are two main disadvantage to Soloshot’s solution. Firstly, the user is required to the wear a tag for the mount to follow. This would involve the user to attach onto themselves somehow or place it in their clothing. Using the track tags also raise an additional point of failure. If the tag failures to send signals to the mount then the solution no longer satisfies the problem statement. Secondly, the cost of the technology is very high with respect to normal camera expenses. With a mount that costs \$500, the technology is more geared towards professionals rather than amateurs. 

\subsection{Swivl}
Swivl is a video capturing platform built for mobile platforms such as Android and iOS. Swivl’s solution to eliminating the cameraman problem is to utilize mobile devices and placing them on their custom mount. For the mount to track a user’s motion, the user will hold onto a marker that the mount will communicate with bluetooth signals. The mount then tilts and rotates accordingly to how the marker is moved.
\par
Similar to the solution provided by Soloshot, Swivl requires the need of a marker that communicates with the mount to track user motions. Another disadvantage to Swivl is that the marker must be within line-of-sight. If the marker goes behind an object than the mount will not be able to detect motion.

\subsection{AlMe}
AIMe aims your camera and keeps the subject framed. It uses an IR emitter that can be attached to any object that you want to track called EmIT. It pulses an IR signal differently than than what occurs naturally indoors or outdoors, so it's able to detect a unique pulsing pattern, lock on to it, and follow it, through tilting and swiveling the motors. AIMe is able to move up to 100 times per second and can be visible to over 100 feet away indoors and over 300 feet away outdoors. It's able to work with any small camera that has compatible mounting. Hardware wise, AIMe has double-sealed steel ball bearings that allow it to accomodate thousands of pounds of force, and is able to react to being dropped or jolted.

\section{Our Solution}
Our solution would address any issues that the above solutions have as well as use a different method of tracking. The above solutions use a physical device to track the position of the subject for filming, but we would like to research the abilities of a solution that uses image processing to track the subject. This would hopefully be a cheaper solution than the existing ones and involve writing our own algorithm for the tracking. Another important improvement would be more fluid movement as current solutions seem to have very jerky movement when following the subject. We hope to also allow for multiple targets. A challenge will be to use image processing to track but having the user go out of the line of sight. This has not been solved yet.

\end{document}
